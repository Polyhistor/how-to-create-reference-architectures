\documentclass[12pt,a4paper]{article}
\usepackage[utf8]{inputenc}
\usepackage{amsmath}
\usepackage{amsfonts}
\usepackage{amssymb}
\usepackage{graphicx}
\usepackage{hyperref}
\usepackage{natbib}

\title{How to Create a Reference Architecture: Addressing Methodological Gaps and Challenges}
\author{Pouya Ataei}
\date{}

\begin{document}

\maketitle

\begin{abstract}
This paper addresses the methodological gaps in creating reference architectures, focusing on the challenges of developing kernel and design theories, translating these theories into architectural constructs, and evaluating the resulting architectures. We propose a comprehensive framework to guide researchers and practitioners through the process of reference architecture creation, addressing key limitations in current approaches.
\end{abstract}

\section{Introduction}
\label{sec:introduction}
Reference architectures play a crucial role in guiding the development of complex systems...

\section{Current State of Reference Architecture Development}
\label{sec:current-state}
Existing methodologies for creating reference architectures, such as Galster and Avgeriou's empirically-grounded approach \citep{Galster2011}, provide valuable insights but have several limitations...

\section{Challenges in Reference Architecture Creation}
\label{sec:challenges}

\subsection{Developing Kernel and Design Theories}
\label{subsec:theories}
One of the primary challenges in creating reference architectures is the development of robust kernel and design theories...

\subsection{Translating Theories into Architectural Constructs}
\label{subsec:translation}
Once theories are developed, translating them into concrete architectural elements presents its own set of challenges...

\subsection{Evaluation Difficulties}
\label{subsec:evaluation}
Existing methods for evaluating reference architectures often fall short in assessing both theoretical validity and practical applicability...

\subsection{Lack of Standardization in Presentation}
\label{subsec:standardization}
Another significant challenge in the field of reference architecture creation is the lack of standardization in how these architectures are presented and documented. This inconsistency makes it difficult for stakeholders to compare, evaluate, and implement different reference architectures effectively. The absence of a unified presentation format can lead to misinterpretations, hinder knowledge transfer, and impede the widespread adoption of reference architectures across various domains. Furthermore, this lack of standardization complicates the process of validating and refining existing architectures, as well as integrating insights from multiple sources to create more comprehensive and robust reference models.

\subsection{Theory Development Phase}
\label{subsec:theory-dev}
Our proposed framework begins with a systematic approach to theory development...

\subsection{Architectural Construction Phase}
\label{subsec:arch-construction}
The next phase involves a methodical translation of theories into architectural components...

\subsection{Iterative Refinement Process}
\label{subsec:refinement}
To ensure the ongoing relevance and effectiveness of the reference architecture, we propose an iterative refinement process...

\subsection{Comprehensive Evaluation Framework}
\label{subsec:eval-framework}
Our framework includes a multi-faceted approach to evaluation...

\section{Case Study: Application of the Proposed Framework}
\label{sec:case-study}
To demonstrate the application of our proposed framework, we present a case study based on the development of the Metamycelium reference architecture...

\section{Discussion}
\label{sec:discussion}
The proposed framework addresses several key limitations of existing methodologies...

\section{Conclusion}
\label{sec:conclusion}
This paper has presented a comprehensive framework for creating reference architectures...

\bibliographystyle{apalike}
\bibliography{references}

\end{document}